\documentclass{ctexart}  

%\usepackage{fontspec}               % Load fontspec  
%\usepackage{unicode-math}           % Load unicode-math  
%\setmathfont{Latin Modern Math}     % Set the math font to Latin Modern Math  
%\setmathfont{STIX Math}            % Set the math font to STIX Math



\usepackage{amsmath}                % AMS math features  
\usepackage{amssymb}                % AMS symbols  
\usepackage{calrsfs}                % Calligraphic script fonts  
\usepackage{bm}                     % Bold math symbols  
\usepackage{stmaryrd}               % Load stmaryrd after unicode-math  

\begin{document}
设关系$\sim$是集合$X$上的一个等价关系,则集合
\[\left\llbracket x\right\rrbracket_\sim \equiv \left\{ y \mid y \in X \wedge \left( \exists x \in X, y \sim x \right) \right\}\]
称$x$关于$\sim$的\emph{等价类(equivalent class)}\footnote{“类”与“集合”在概念上无实质区别。}。

$X$的元素关于$\sim$的所有等价类的集合,记作$X/\sim$\footnote{注意与相对补集的符号相区别。},称为集合$X$在等价关系$\sim$下的\emph{商集(quotient set)}\footnote{正式地,
    \[
        X/\sim \equiv \left\{ A \in \mathcal{P}(X) \mid \forall x \left( x \in A \wedge \left\llbracket x \right\rrbracket_{\sim} = A \right) \right\}
    \]
}。

$\bm{\mu}$

\end{document}