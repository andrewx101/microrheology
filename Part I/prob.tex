\documentclass[main.tex]{subfiles}
\begin{document}
本章是在华南理工大学编写的概率论与数理统计教材\cite{何春雄2012}(以下简称本科教材)基础上的补充。



\section{正态分布的随机向量}
在本科教材的例3.3.4、例3.3.5已经介绍过$n$维正态分布,在这里我们正式地给出定义。设$\mathbf{X}=\left(X_1,\dots,X_n\right)^\intercal$是一个随机向量。如果存在一个向量$\boldsymbol{\mu}=\left(\mu_1,\cdots,\mu_n\right)$和一个对称、半正定$n\times n$矩阵$\boldsymbol{\Sigma}$,使得$\mathbf{X}$的特征函数是
\[\varphi_{\mathbf{X}}\left(\mathbf{u}\right)=\exp\left(\imagI\mathbf{u}^\intercal\boldsymbol{\mu}-\frac{1}{2}\mathbf{u}^\intercal\boldsymbol{\Sigma}\mathbf{u}\right)\]
则称随机向量$\mathbf{X}$满足$n$维正态分布,记作$\mathbf{X}\sim\mathcal{N}\left(\boldsymbol{\mu},\boldsymbol{\Sigma}\right)$。如果$\boldsymbol{\Sigma}$是非奇异的,就称这种情况为非退化情况(nondegenerate case),反之则称为退化情况(degenerate case)。在退化情况下,$\mathbf{X}$没有普通意义的概率密度函数。我们只讨论非退化情况并不再说明,此时要求定义中的$\boldsymbol{\Sigma}$是正定的。

从上述定义可以推出,$\mathbf{X}$的概率密度函数、期望、和协方差矩阵如下
\begin{align*}
    f_\mathbf{X}\left(\mathbf{x}\right) & =\left[\left(2 \pi\right)^n\mathrm{det}\boldsymbol{\Sigma}\right]^{1/2}\exp\left[-\frac{1}{2}\left(\mathbf{x}-\boldsymbol{\mu}\right)^\intercal\boldsymbol{\Sigma}^{-1}\left(\mathbf{x}-\boldsymbol{\mu}\right)\right] \\
    \mu_i                               & =\mathrm{E}\left[X_i\right],\quad \Sigma_{ij}=\mathrm{Cov}\left[X_i,X_j\right],\quad i,j=1,\cdots,n
\end{align*}
其中$\mathbf{x}=\left(x_1,\cdots,x_n\right)^\intercal$。$f_\mathbf{X}\left(\mathbf{x}\right)$也称作随机向量$\mathbf{X}$的联合概率密度函数,它就是$n$个随机变量$X_1,\cdots,X_n$的联合概率密度函数$f_{X_1,\cdots,X_n}\left(x_1,\cdots,x_n\right)$。

% 这里,学生是否已经习惯向量函数的表示方式?

\begin{theorem}
    若$\mathbf{X}\sim\mathcal{N}\left(\boldsymbol{\mu},\boldsymbol{\Sigma}\right)$则$\mathbf{Ax}+\mathbf{b}\sim\mathcal{N}\left(\mathbf{A}\boldsymbol{\mu}+\mathbf{b},\mathbf{A}\boldsymbol{\Sigma}\mathbf{A}^\intercal\right)$
\end{theorem}


高斯分布的随机向量的线性(仿射)变换结果还是一个高斯分布的随机向量。具体地,
\end{document}